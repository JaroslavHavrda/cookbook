\documentclass[12pt,a4paper]{article}
\usepackage[czech]{babel}
\usepackage[utf8]{inputenc}
\usepackage[newpage]{cooking}
\usepackage{textcomp}
\pagestyle{empty}
\renewcommand{\sidedish}[1]{\item {#1}}

\begin{document}

\vspace*{\fill}
{\Huge\it
\centerline{Polévky}
\centerline{Kuře}
\centerline{Kachna}
\centerline{Králík}
\centerline{Ryby}
\centerline{Vepřové}
\centerline{Mleté maso}
\centerline{Ostatní maso}
\centerline{Bezmasé}
\centerline{Brambory}
\centerline{Omáčky}
\centerline{Přílohy}
\centerline{Saláty}
\centerline{Svačinky}
\centerline{Pečivo}
\centerline{Dezerty}
\centerline{Cukroví}
\centerline{Nápoje}
\centerline{Těsta}
}
\vspace*{\fill}
\newpage

\vspace*{\fill}
\hfill {\Huge\it Polévky}\hfill
\vspace*{\fill}
\newpage

\begin{recipe}{Arménská čočková polévka}
  \ingredient{1,5 hrnku čočky} propláchnout a nechat namočené pár hodin, poté scedit
  \ingredient{hrnek cibule}
  \ingredient{1/2 hrnku mrkve}
  \ingredient{1/2 hrnku celeru}
  \ingredient{3 hrnky lilku (1 ks)}
  \ingredient{hrnek zelené papriky}
  \ingredient{3 - 6 stroužků česneku} opékat na oleji do zlatohněda
  \ingredient{6 hrnků kuřecího vývaru (asi 2 l)}
  \ingredient{1,5 hrnku rajčat}
  \ingredient{100 g sušených meruněk}
  \ingredient{1/4 lžičky skořice}
  \ingredient{1/4 lžičky nového koření}
  \ingredient{1 lžíce sladké papriky}
  \ingredient{1,5 lžičky soli}
  \ingredient{1/4 lžičky kayenského pepře} smíchat v hrnci, přidat čočku a zeleninovou směs, vařit, dokud čočka nezměkne
  \ingredient{3 lžíce posekané zelené petrželky}
  \ingredient{lžíce posekané máty} přidat na závěr vaření
\end{recipe}
\newpage

\begin{recipe}{Boží vločková polévka}
   \ingredient{70g másla} rozpustit na pánvi
   \ingredient{1 cibule} osmažit
   \ingredient{5 lžic ovesných vloček} přidat, osmažit do křupava
   \ingredient{2 vejce} přidat, usmažit
   \ingredient{sůl}
   \ingredient{voda} přidat, nechat projít varem
   \ingredient{bylinky} přidat
\end{recipe}
\newpage

\begin{recipe}{Cuketová polévka s hráškem a mátou}
  \ingredient{600 g cukety} omýt, nakrájet na kolečka a vařit v hrnci s 200 ml vody, uvařit do měkka, rozmixovat, osolit
  \ingredient{400 g hrášku} uvařit v osolené vodě, poté slít vodu do misky, hrášek přimíchat k cuketě a případně poředit vodou z misky
  \ingredient{2 lžíce strouhaného parmazánu}
  \ingredient{hrst najemno pokrájené čerstvé máty} přidat do polévky, promíchat a podávat s čerstvým pečivem
\end{recipe}
\newpage

\begin{recipe}{Hovězí polévka s nudlemi (1 porce)}
  \ingredient{ 50 g zeleniny } nakrájet na kostičky, uvařit
  \ingredient{ hovězí vývar } přidat
  \ingredient{ 15 g nudlí } uvařit a přidat
  \ingredient{ sůl } přidat
  \ingredient{ petrželka } přidat
\end{recipe}
\newpage

\begin{recipe}{Květáková polévka}
  \ingredient{1l vývaru}
  \ingredient{200g květáku} nakrájet, povařit ve vývaru
  \ingredient{40g másla}
  \ingredient{40g hladké mouky} udělat světlou jíšku, vmíchat k vývaru, ohřát do varu
  \ingredient{120g mléka} přidat
  \ingredient{4 špetky muškátového květu}
  \ingredient{4 špetky petržele} přidat
\end{recipe}
\newpage

\begin{recipe}{Minestra Marinata}
  \ingredient{voda}
  \ingredient{sůl} dát vařit
  \ingredient{400g kapusty} nakrájet nahrubo, 5 minut blanšírovat, ocedit
  \ingredient{2l vody}
  \ingredient{500g uzené šunky} do vody
  \ingredient{1 velká cibule}
  \ingredient{1 list řapíkatého celeru}
  \ingredient{1 velká mrkev}
  \ingredient{1 brambora} na hrubo nakrájet, do vody
  \ingredient{8 cherry rajčat}
  \ingredient{1/2 lžičky fenyklových semínek}
  \ingredient{8 kuliček pepře} do vody, vařit hodinu
  \ingredient{250g salátu Batavia} do vody, maso nakrájet na kostičky, přidat kapustu
  \ingredient{sůl} dochutit, vařit 40 minut
  \ingredient{olivový olej} pokapat porci na talíři.
\end{recipe}
\newpage

\begin{recipe}{Ovocná polévka}
  \ingredient{ovoce} rozmixovat
  \ingredient{voda} přidat, vařit
  \ingredient{mléko}
  \ingredient{hladká mouka} do mléka, to do polévky
  \ingredient{cukr} do polévky
  \ingredient{ovoce} nakrájet do polévky
\end{recipe}
\newpage

\begin{recipe}{Špenátová polévka se sýrovými šneky}
  \ingredient{listové těsto} rozvélet
  \ingredient{kysaná smetana} potřít těsto
  \ingredient{niva} nastrouhat na těsto
  \ingredient{oříšky} nasekat, na těsto
  \ingredient{špenátové listy} na těsto, zarolovat, nakrájet
  \ingredient{vajíčko} rozšlehat, potřít šneky, dát péct do horké trouby
  \ingredient{2 lžíce másla} rozpustit
  \ingredient{50g mouky} přidat, udělat jíšku
  \ingredient{750ml vody} přidat, rozmíchat do hladka
  \ingredient{250ml mléka} přidat
  \ingredient{200ml 12 procentní smetany} přidat, uvést do varu
  \ingredient{350g špenátu} nakrájet, přidat, uvařit do měkka, rozmixovat
  \ingredient{stroužek česneku} proslisovat, přidat
  \ingredient{sůl} přidat
  \ingredient{pepř} přidat
\end{recipe}
\newpage

\vspace*{\fill}
\hfill {\Huge\it Kuře}\hfill
\vspace*{\fill}
\newpage

\begin{recipe}{Kuře s jablky}
  \ingredient{kuře}
  \ingredient{sůl} na kuře
  \ingredient{jablka} na plátky, do kuřete
  \ingredient{máslo} na pekáč, na něj kuře
  \ingredient{pomerančová šťáva} pokapat kuře, upéct.
  \sidedish{brambory, salát} 
\end{recipe}
\newpage

\begin{recipe}{Kuře s medovou kůrčičkou}
  \ingredient{kuře}
  \ingredient{sůl}
  \ingredient{olej} na kuře, upéct
  \ingredient{1+ 1/2 lžíce medu}
  \ingredient{1 + 1/2 lžíce hořcice} na kuře, dopéct 20 minut
\end{recipe}
\newpage

\begin{recipe}{Kuřecí prsa v semínkové krustě}
  \ingredient{600 g kuřecích prsou} naříznout, aby vznikla kapsa
  \ingredient{30 g parmazánu} nastrouhat
  \ingredient{60 g sýru feta} nastrouhat
  \ingredient{stroužek česneku} utřít
  \ingredient{snítka bazalky} posekat
  \ingredient{snítka oregána} posekat
  \ingredient{špetka mleté slaté papriky}
  \ingredient{špetka mletého koriandru}
  \ingredient{špetka mletého muškétového oříšku}
  \ingredient{60 g másla} přimíchat sýry, česnek, bylinky a koření, vetřít do kapes, otvor spíchnout jehlou, prsa položit na pečicí plech potřený olejem
  \ingredient{1 vejce} rozšlehat
  \ingredient{20 g slunečnicových semínek}
  \ingredient{20 g dýňových semínek}
  \ingredient{30 g piniových ořechů} vmíchat do vejce, hmotou potřít kuřecí prsa, péct zvolna při 140°C asi 20-25 minut
\end{recipe}
\newpage

\begin{recipe}{Přírodní maso s cuketou}
  \ingredient{4 kuřecí (nebo jiné) řízky} naklepat, osoli, opepřit, na oleji zprudka z obou stran osmažit
  \ingredient{400 g cukety} nakrájet na kolečka, orestovat ve výpeku z masa
  \ingredient{sojová omáčka} dle chuti ještě na pánvi pokapat na cukety, plátky masa dát na talíř a na ně navršít cukety
\end{recipe}
\newpage

\vspace*{\fill}
\hfill {\Huge\it Kachna}\hfill
\vspace*{\fill}
\newpage

\begin{recipe}{Kachna pomalu pečená}
  \ingredient{1 kachna}
  \ingredient{sůl}
  \ingredient{kmín} kachnu osolit a okmínovat z obou stran (i uvnitř), do pekáče na záda, přiklopit, péct na 120°C bez otevření trouby 4 hodiny, pak sejmout víko, zvednout teplotu na 150-180°C a péct do křupava dalších 20-30 minut 
\end{recipe}
\newpage

\vspace*{\fill}
\hfill {\Huge\it Králík}\hfill
\vspace*{\fill}
\newpage

\begin{recipe}{Dušený králík s rozmarýnem a bílým vínem}
  \ingredient{králík} opláchnout, důkladně osušit, očistit, případně naporcovat
  \ingredient{sůl, pepř} ochutit králíka
  \ingredient{hladká mouka} obalit kousky králíka ze všech stran
  \ingredient{olivový olej} nalít do hrnce, zprudka opéct králíka do zlatohněda
  \ingredient{6 stroužků česneku} rozmačkat, přidat do hrnce
  \ingredient{4 větvičky rozmarýnu} lístky přidat do hrnce
  \ingredient{200 ml suchého bílého vína} podlít králíka a počkat, dokud se alkohol neodpaří
  \ingredient{500 ml kuřecího nebo zeleninového vývaru} zalít králíka asi do 2/3, dusit asi 40 minut pod pokličkou na mírném ohni, dokud maso nezměkne
  \ingredient{2 lžíce másla} zjemnit omáčku
  \ingredient{hrst petrželky} nasekat, přidat na závěr
  \ingredient{4 stroužky česneku} nakrájet nadrobno
  \ingredient{2 lžíce másla} rozpustit na pánvi
  \ingredient{trocha olivového oleje} přidat k máslu, opéct česnek dozlatova
  \ingredient{250 g listového špenátu} omýt a osušit, osolit, opepřit a nechat chvíli spařit na pánvi s česnekem
  \ingredient{bílý chléb} opéct na olivovém oleji do zlatohněda, na talíř servírovat chléb, na něj špenát, na to králíka a polít omáčkou
\end{recipe}
\newpage

\begin{recipe}{Pečený králík na hořčici}
  \ingredient{králičí zadek}
  \ingredient{hořčice} namazat na králíka, nechat den odležet, pak otřít
  \ingredient{tuk} na pánev
  \ingredient{zelenina} podusit, přidat králíka, péct, vyndat králíka
  \ingredient{mouka} zahustit omáčku
  \ingredient{hořčice} přidat podle chuti
  \ingredient{kyselá okurka} přidat podle chuti
\end{recipe}
\newpage

\vspace*{\fill}
\hfill {\Huge\it Ryby}\hfill
\vspace*{\fill}
\newpage

\begin{recipe}{Losos na cuketovém ragů}
  \ingredient{česnek} nakrájet na plátky
  \ingredient{olivový olej} orestovat česnek
  \ingredient{cuketa} nakrájet na půlkolečka, přidat k česneku, osolit a opepřit, nechat chvíli opékat
  \ingredient{bílé víno} podlít cuketu, nechat vyvařit
  \ingredient{rajčátka} rozkrojit, přidat k cuketě
  \ingredient{bazalka} nasekat, přidat k cuketě a rajčatům a ještě chvíli opékat
  \ingredient{máslo} zjemnit ragů
  \ingredient{losos}
  \ingredient{máslo}
  \ingredient{česnek}
  \ingredient{tymián} zapéct na 200°C asi na 15 - 20 minut, na talíř dát ragů a na něj lososa
\end{recipe}
\newpage

\begin{recipe}{Losos na špenátové peřince}
  \ingredient{losos} očistit, osolit, mírně opepřit
  \ingredient{olivový olej} mírně osmažit lososa, přendat do zapékací mísy
  \ingredient{ořechy} nasekat na kousky
  \ingredient{moučkový cukr} přidat k ořechům v poměru cca 5:1 (víc ořechů), směsí posypat lososa, dát zapéct do trouby na 180°C asi na 15 - 20 minut
  \ingredient{200 g špenátu} opláchnout
  \ingredient{3 stroužky česneku} nakrájet na plátky
  \ingredient{jarní cibulka} nakrájet na kolečka, spolu s česnekem osmahnout na pánvi, přidat špenát, lehce podlít
  \ingredient{1 jablko} nakrájet na tenké plátky, přidat ke špenátu, osolit a podusit, nechat odpařit vodu, na talíř dát špenát a na něj lososa
\end{recipe}
\newpage

\begin{recipe}{Treska zapečená s rajčaty a mozzarellou}
  \ingredient{olivový olej} vymazat pekáček
  \ingredient{treska} filety osolit, dát do pekáčku
  \ingredient{česnek} prolisovat, potřít tresku
  \ingredient{rajčata} nakrájet na kolečka, poklást na tresku
  \ingredient{mandle} posekat, posypat rajčata, dát péct na 200°C
  \ingredient{mozzarella} nakrájet na kolečka
  \ingredient{bazalka} před dopečením spolu se sýrem přidáme na tresku
\end{recipe}
\newpage

\vspace*{\fill}
\hfill {\Huge\it Vepřové}\hfill
\vspace*{\fill}
\newpage

\begin{recipe}{Chřest s holandskou omáčkou}
  \ingredient{1kg malých vařených brambor} rozpůlit
  \ingredient{sůl} na brambory
  \ingredient{olej} na pánev, osmažit brambory (jen s nimi občas potřást)
  \ingredient{600g masa}
  \ingredient{sůl} na maso
  \ingredient{pepř} na maso
  \ingredient{olej} na pánev, restovat maso, pak ho dát do trouby
  \ingredient{3 žloutky} do misky ve vodní lázni
  \ingredient{2 lžíce vody}
  \ingredient{lžíce hořčice} rozmíchat, šlehat metličkou, dokud se nenapění a nezesvětlají (tři až pět minut)
  \ingredient{100g másla} nakrájet na plátky, postupně zašlehat
  \ingredient{sůl}
  \ingredient{citronová šťáva} dochutit omáčku
  \ingredient{12ks blanšítovaného chřestu} prohřát ve vodě
\end{recipe}
\newpage

\begin{recipe}{Jarní nádivka se špenátem a pórkem}
  \ingredient{500g vepřové plece} uvařit do měkka
  \ingredient{200g uzeného masa} uvařit do měkka
  \ingredient{6 rohlíků} nakrájet na kostičky
  \ingredient{2 pórky} očistit, nakrájet na kolečka
  \ingredient{lžíce másla} rozpustit na pánvi, podusit na ní pórek
  \ingredient{sůl} na pórek
  \ingredient{6 vajec} oddělit žloutky a bílky, z bílků sníh
  \ingredient{300ml mléka} do žloutků
  \ingredient{sůl} do mléka, nalít na rohlíky, přidat maso nakrájené na kostičky
  \ingredient{100g špenátových listů} přidat
  \ingredient{pepř} přidat, přidat sníh, dát do pekáčku, péct při 180 stupních dozlatova
\end{recipe}
\newpage

\begin{recipe}{Kotleta excelsior}
  \ingredient{kotlety} nařiznout tukové vrstvy, naklepat u kosti
  \ingredient{sůl} na kotlety
  \ingredient{sádlo} na pánev, prudce osmažit kotlety
  \ingredient{cibulka} osmažit
  \ingredient{jablka} osmažit s cibulí
  \ingredient{jablečný mošt} přidat
  \ingredient{máslo} přidat, přidat kotlety, péct při 160, pak při 180 stupních celsia.
\end{recipe}
\newpage

\begin{recipe}{Statkářská bída}
  \ingredient{2 lžíce másla} rozpustit na pánvi
  \ingredient{svazeček mladých cibulí} osmahnout na másle
  \ingredient{60 g hladké mouky} zaprášit cibuli, upražit
  \ingredient{1/4 l smetany} přidat
  \ingredient{1/4 l kysané smetany} přidat, za stálého míchání povařit
  \ingredient{sůl} přidat
  \ingredient{pepř} přidat
  \ingredient{bylinky} přidat, zamíchat
  \ingredient{1 lžíce olivového oleje} na pánev
  \ingredient{200 g uzeného masa} na kostičky, na pánev
  \ingredient{150 g vařených brambor} nakrájet, osmažit s masem, vmíchat do omáčky
  \ingredient{6 vajec na tvrdo} nakrájet, vmíchat do omáčky
\end{recipe}
\newpage

\vspace*{\fill}
\hfill {\Huge\it Mleté maso}\hfill
\vspace*{\fill}
\newpage

\begin{recipe}{Plněné cibule}
  \ingredient{300g mletého masa}
  \ingredient{100g uzeného boku nabo krkovice} najemno nakrájet, nandat do masa
  \ingredient{100ml mléka} do masa
  \ingredient{česnek} utřít do masa
  \ingredient{sůl}
  \ingredient{kmín}
  \ingredient{majoránka} do masa
  \ingredient{600g cibule} rozkrojit, vydlabat, naplnit masem
  \ingredient{sádlo} do pekáče, přidat cibule (plné i vnitřky), upéct
\end{recipe}
\newpage

\begin{recipe}{Plněná jablka}
  \ingredient{mleté maso (750g)}
  \ingredient{slanina} nakrájet na kostičky, do masa
  \ingredient{cibule} nastrouhat do masa
  \ingredient{křen}
  \ingredient{sůl}
  \ingredient{pepř} vmíchat
  \ingredient{jablka (6)} vydlabat, naplnit masem, zabalit do alobalu, péct půl až třičtvrtě hodiny
  \ingredient{smetana} zalít jablka, dopéct
\end{recipe}
\newpage

\vspace*{\fill}
\hfill {\Huge\it Ostatní maso}\hfill
\vspace*{\fill}
\newpage

\vspace*{\fill}
\hfill {\Huge\it Bezmasé}\hfill
\vspace*{\fill}
\newpage

\begin{recipe}{Aglio olio e peperoncino}
  \ingredient{400 g špaget} uvařit na skus v dobře osolené vodě
  \ingredient{1 dcl olivového oleje} nalít na větší studenou pánev
  \ingredient{16 stroužků česneku}
  \ingredient{2 čerstvé feferonky} rozehřát na oleji
  \ingredient{hrst nasekané petrželové natě} přidat na pánev, trošku prohřát, přidat špagety s trochou vody, ve které se vařily, a promíchat
\end{recipe}
\newpage

\begin{recipe}{Amarantový kotlík}
  \ingredient{250 g amarantu} vařit v 500 ml vody 25 - 30 minut a nechat dojít pod pokličkou
  \ingredient{4 lžíce slunečnicového oleje} rozehřát na hluboké pánvi
  \ingredient{1 stonek řapíkatého celeru}
  \ingredient{1 červená paprika}
  \ingredient{snítka tymiánu} osmahnout na oleji
  \ingredient{300 ml (sojového) mléka} zalít zeleninu
  \ingredient{100 g pórku} přidat k zelenině a krátce podusit
  \ingredient{150 g sterilované kukuřice}
  \ingredient{hrst čerstvé petrželky} spolu s amarantem přimíchat k zelenině
  \ingredient{4 lžíce (sojové) smetany} vmíchat pro zjemnění
\end{recipe}
\newpage

\begin{recipe}{Bramborové šišky}
  \ingredient{voda} dát vařit
  \ingredient{600g vařených brambor} najemno nastrouhat
  \ingredient{200g hrubé mouky}
  \ingredient{1 vejce} zadělat těsto, vyválet váleček, vařit čtyři minuty
  \ingredient{mletý mák}
  \ingredient{moučkový cukr} nasypat na šišky
  \ingredient{máslo} rozhřát a nalít na šišky
\end{recipe}
\newpage

\begin{recipe}{Čočkový prejt}
  \ingredient{250-300 g čočky} (namočené a propláchnuté)
  \ingredient{3 zrnka nového koření} vaříme do poloměkka
  \ingredient{150 g jáhel} dvakrát spaříme, přidáme k čočce a dovaříme do měkka
  \ingredient{60 g tuku}
  \ingredient{česnek}
  \ingredient{cibule}
  \ingredient{sůl}
  \ingredient{majoránka} přidáme k čočce a ve vymazaném pekáčku dáme krátce zapéct
\end{recipe}
\newpage

\begin{recipe}{Falafel}
  \ingredient{250 g cizrny} namočit nejlépe na 24 hodin, propláchnout, zcedit, rozmixovat na pyré, dát do mísy, část nechat v mixéru
  \ingredient{hrst koriandru}
  \ingredient{hrst petrželky}
  \ingredient{chilli paprička}
  \ingredient{3 stroužky česneku} posekat, přidat do mixéru k části cizrny a společně dobře rozmixovat, přidat ke zbytku cizrny
  \ingredient{2 lžíce mouky}
  \ingredient{3/4 lžičky jedlé sody}
  \ingredient{1 lžička mletého kmínu}
  \ingredient{1 lžička drcených koriandrových semínek}
  \ingredient{1,5 lžičky soli}
  \ingredient{pepř dle chuti} přidat do mísy, dobře zpracovat lžící, v případě potřeby přidat pár kapek vody, dělat malé kuličky  (25 - 30 ks)
  \ingredient{hodně oleje} rozehřát na střední až vysokou teplotu, jednotlivé kuličky dát lžící do oleje, smažit do hněda a pak nechat na ubrousku odkapat přebytečný olej
\end{recipe}
\newpage

\begin{recipe}{Involuti di zuchini}
  \ingredient{2-3 cukety} nakrájet na plátky
  \ingredient{sůl} posypat cukety, nechat 40 min, pak osušit
  \ingredient{olej} na plech
  \ingredient{stroužek česneku} nastrouhat
  \ingredient{4 lžíce olivového oleje} přidat
  \ingredient{pepř mletý} přidat, namazat cukety, péct 10-15 minut, otočit cukety, namezat, péct minutu, nechat vychladnout
  \ingredient{100g gorgonzoly} na cukety
  \ingredient{sušená rajčata} na cukety
  \ingredient{bazalka} na cukety, zavinout
\end{recipe}
\newpage

\begin{recipe}{Jablkové nudle}
  \ingredient{250g tenkých nudlí} uvařit do měkka
  \ingredient{60g másla}
  \ingredient{60g cukru}
  \ingredient{3 žloutky} utřít
  \ingredient{citronová kůra} vmíchat
  \ingredient{3 bílky} ušlehat sníh, vníchat nudle
  \ingredient{500g jablek} oloupat, nakrájet na plátky, vmí\-chat
  \ingredient{tuk, strouhanka} vysypat formu, péct
\end{recipe}
\newpage

\begin{recipe}{Jáhlové placky}
  \ingredient{1 hrnek jáhel} (spařených)
  \ingredient{500 ml vody}
  \ingredient{1 bujón} povaříme, dokud se voda neodpaří
  \ingredient{3 vejce}
  \ingredient{1 cibule} (najemno nakrájená)
  \ingredient{1 paprika} (najemno nakrájená)
  \ingredient{česnek}
  \ingredient{2 lžičky majoránky}
  \ingredient{2 lžičky soli} přidáme k jáhlám a dobře promícháme
  \ingredient{5 lžic hladké mouky}
  \ingredient{1/2 hrnku strouhanky} postupně přimícháváme k jáhlám, znovu dobře promícháme, tvoříme placky a smažíme z obou stran
\end{recipe}
\newpage

\begin{recipe}{Koláč s tvarůžky}
  \ingredient{250 - 300 g listového těsta} vyválet do formy o průměru přibližně 31 cm (na pečicí papír)
  \ingredient{200 g tvarůžků} nastrouhat nahrubo
  \ingredient{100 g anglické slaniny} nakrájet najemno
  \ingredient{50 g pórku} nakrájet na tenčí kolečka
  \ingredient{2 vejce} oddělit bílky a žloutky, k žloutkům přimíchat tvarůžky, slaninu a pórek, z bílků ušlehat sníh a zamíchat do hmoty, tu rozprostřít na listové těsto
  \ingredient{paprika} nakrájet a poklást na koláč, péct na 180°C asi 25 minut
\end{recipe}
\newpage

\begin{recipe}{Langoše/ těsto na pizzu}
  \ingredient{kostka droždí}
  \ingredient{1 dcl mléka}
  \ingredient{trocha cukru} zadělat kvásek
  \ingredient{1/2 kg hladké mouky}
  \ingredient{2,5 dcl mléka}
  \ingredient{lžička soli} přidat kvásek a nechat kynout, dělat placky, smažit v oleji
  \ingredient{česnek, sýr, tatarka, kečup...} dle chuti
\end{recipe}
\newpage

\begin{recipe}{Ratatouille}
  \ingredient{6 barevných paprik}
  \ingredient{2 cukety}
  \ingredient{1 lilek} 
  \ingredient{provensálské koření}
  \ingredient{olivový olej} jednotlivé druhy zeleniny pokrájet a každý druh zvlášť orestovat na olivovém oleji s provensálským kořením
  \ingredient{1 cibule} nakrájet a orestovat na oleji
  \ingredient{4 rajčata} nakrájet, přidat k cibuli
  \ingredient{sůl} posolit rajčata, přidat provensálské koření a orestovat, všechnu orestovanou zeleninu smíchat ve velkém hrnci a pokračovat ve vaření
  \ingredient{snítka rozmarýnu}
  \ingredient{snítka tymiánu}
  \ingredient{bílé víno} přidat k zelenině, podle chuti dosolit a zeleninu téměř rozvařit, podávat s vaječnou omeletou, s~toasty, nebo jako přílohu
\end{recipe}
\newpage

\begin{recipe}{Rýžový nákyp}
  \ingredient{400g rýže} spařit, dát do vymaštěného pekáče
  \ingredient{máslo} na plátky, dát na rýži
  \ingredient{1.5 l mléka} nalít
  \ingredient{trochu soli} přidat
  \ingredient{cukr} přidat, pomalu péct do zhoustnutí
  \ingredient{ovoce} nakrájet, přidat, dopéct
  \ingredient{skořicový cukr} posypat.
\end{recipe}
\newpage

\begin{recipe}{Žampiony plněné špenátem}
  \ingredient{20 žampionů} očistit, nožičky nasekat na jemné kousky
  \ingredient{lžíce másla} rozehřát
  \ingredient{2 stroužky česneku} nakrájet na plátky
  \ingredient{1 cibule} jemně nasekat, jemně orestovat na másle s česnekem a nožičkami žampionů
  \ingredient{30 g špenátu} nasekat, přidat na pánev a povařit, až se voda vypaří
  \ingredient{4 lžíce strouhanky}
  \ingredient{60 g strouhaného parmazánu} zamíchat do směsi
  \ingredient{sůl, pepř, bazalka a oregano} přidat dle chutii, dobře promíchat, nechat chvilku prohřát, směsí naplnit hlavičky žampionů, rozložit do olejem vymaštěného pekáčku, péct asi 15 minut na 220°C
\end{recipe}
\newpage

\vspace*{\fill}
\hfill {\Huge\it Brambory}\hfill
\vspace*{\fill}
\newpage

\begin{recipe}{Bramborová pizza}
  \ingredient{2 -- 3 kg brambor}
  \ingredient{vajíčka}
  \ingredient{česnek}
  \ingredient{sůl}
  \ingredient{pepř}
  \ingredient{majoránka}
  \ingredient{hladká mouka} udělat bramborákové těsto
  \ingredient{olej} vymazat plech, nalít těsto
  \ingredient{šunka}
  \ingredient{paprika}
  \ingredient{rajče} nakrájet na těsto, dát péct
  \ingredient{sýr} nakrájet na upečené, dopéct do roztavení sýra
\end{recipe}
\newpage

\begin{recipe}{Smetanové brambory}
  \ingredient{brambory} nakrájet na kolečka, půlku dát do pekáčku
  \ingredient{cibule} nakrájet na kolečka, dát do pekáčku, na to zbytek brambor
  \ingredient{smetana}
  \ingredient{bazalka}
  \ingredient{1/4 lžičky skořice} smíchat, půlku nalít na brambory, hodinu péct, dolít zbytek smetany a nechat dojít ve vypnuté troubě
\end{recipe}
\newpage

\vspace*{\fill}
\hfill {\Huge\it Omáčky}\hfill
\vspace*{\fill}
\newpage

\begin{recipe}{Cibulová omáčka}
  \ingredient{maso} uvařit
  \ingredient{1 cibuli}nakrájet, usmažit dozlatova
  \ingredient{2 lžíce hladké mouky}přidat, dosmažit do hněda, zalít vývarem
  \ingredient{2 celé pepře,sůl} přidat, povařit
  \ingredient{2 cibule, slaninu}pokrájet, usmažit do zlatova, přidat do omáčky
\end{recipe}
\newpage

\begin{recipe}{Kuře na paprice}
  \ingredient{cibule} pokrájet
  \ingredient{olej} nalít do hrnce, osmahnout cibuli
  \ingredient{mletá paprika} přisypat k cibuli a dobře promíchat mimo plotýnku
  \ingredient{kuřecí stehna nebo jiné kuřecí maso} přidat k cibuli a osmahnout, zalít vodou, aby bylo maso ponořené
  \ingredient{sůl, pepř} přidat dle chuti, vařit, dokud maso nezměkne, pak vybrat maso z omáčky a obrat
  \ingredient{ovesné vločky} pomlít, zahustit omáčku, povařit ještě 10 minut, pak vrátit maso do omáčky
\end{recipe}
\newpage

\begin{recipe}{Patizonová smetanová omáčka}
  \ingredient{1 cibule} pokrájet najemno
  \ingredient{1 kg patizonů} oloupat, vydlabat semena, nakrájet na kostičky
  \ingredient{50 g másla} na půlce zpěnit cibuli, přidat patizony
  \ingredient{sůl, bílý pepř} ochutit patizony
  \ingredient{250 ml vývaru} zalít patizony, dusit 20 minut
  \ingredient{hladká mouka} s druhou půlkou másla připravit světlou jíšku a zahustit patizony
  \ingredient{smetana} zjemnit omáčku
\end{recipe}
\newpage

\vspace*{\fill}
\hfill {\Huge\it Přílohy}\hfill
\vspace*{\fill}
\newpage

\begin{recipe}{Chřest zapečený se šunkou a sýrem}
  \ingredient{12 blanšírovaných výhonků chřestu}
  \ingredient{6 plátků šunky} obalit chřest
  \ingredient{máslo} vymazat pekáč, naskládat chřest
  \ingredient{sýr} na chřest (nezakrýt hlavičky), péct při 200 stup\-ních pět až osm minut
\end{recipe}
\newpage

\begin{recipe}{Kynuté knedlíky v silikonu (houskové i ovocné)}
  \ingredient{50 ml mléka}
  \ingredient{1/2 kostky droždí}
  \ingredient{trocha cukru} zadělat kvásek
  \ingredient{1 kelímek polohrubé mouky}
  \ingredient{3/4 kelímku mléka}
  \ingredient{2 lžíce oleje}
  \ingredient{1 vejce}
  \ingredient{špetka soli} umíchat těsto, dát vykynout
  \ingredient{nakrájený rohlík/ ovoce} vmíchat, nalít do silikonové formy, dát do mikrovlnky 4 minuty 550 W + 4 minuty 750 W
\end{recipe}
\newpage

\begin{recipe}{Pečená zelenina}
  \ingredient{olej} do pekáčku
  \ingredient{zelenina} očistit a nakrájet do pekáčku
  \ingredient{koření (pepř, kmín k mrkvi, anýz k feniklu, skořici k dýni)} přidat, péct
  \ingredient{med, bylinky} přidat, dopéct
\end{recipe}
\newpage

\vspace*{\fill}
\hfill {\Huge\it Saláty}\hfill
\vspace*{\fill}
\newpage

\begin{recipe}{Salát s opečeným chlebem}
  \ingredient{3 krajíce chleba} nakrájet na hranoly, opéct, nakrá\-jet na kostičky
  \ingredient{1/2 dl vody} ohřát
  \ingredient{3 lžíce cukru} vmíchat do vody
  \ingredient{2 špetky soli} vmíchat do vody
  \ingredient{5 lžic octa} vmíchat do vody, povařit a nechat vychladnout = dresink
  \ingredient{3 lžíce oleje} vmíchat do dresinku
  \ingredient{mladá cuketa, papriky, mladé cibule, rajčata} nakrájet, vmíchat dresink, vmíchat ope\-če\-ný chléb
  \ingredient{balkánský sýr} navrch
\end{recipe}
\newpage

\begin{recipe}{Teplý salát z růžičkové kapusty}
  \ingredient{hrst lískových oříčků} opražit
  \ingredient{lžíce másla} na pánev
  \ingredient{15ks růžičkové kapusty} na pánev, prudce restovat jednu až dvě minuty, aby byla měkká
  \ingredient{100 ml jablečného džusu} přilít
  \ingredient{lžička vinného octa} přilít, míchat, než se tekutina vyvaří, pak salát na talíř
  \ingredient{lžíce másla} na pánev, ohžát v ní ořechy
\end{recipe}
\newpage

\vspace*{\fill}
\hfill {\Huge\it Svačinky}\hfill
\vspace*{\fill}
\newpage

\begin{recipe}{Krůtí toasty}
  \ingredient{1 lžíce pomazánkového másla}
  \ingredient{1 lžička brusinek} přimíchat
  \ingredient{2 plátky toastového chleba} namazat
  \ingredient{2 - 3 plátky krůtí šunky} na chleba
  \ingredient{2 lisky salátu} na šunku, přiklopit druhým chlebem
\end{recipe}
\newpage

\begin{recipe}{Pribináčky}
  \ingredient{1 smetana ke šlehání} ušlehat
  \ingredient{1 tvaroh}
  \ingredient{2 lžičky vanilkového cukru} zašlehat do šlehačky, rozdělit do misek, vychladit
\end{recipe}
\newpage

\vspace*{\fill}
\hfill {\Huge\it Pečivo}\hfill
\vspace*{\fill}
\newpage

\begin{recipe}{Snídaňové housky nehousky}
  \ingredient{3 hrnky hladké mouky}
  \ingredient{1/4 lžičky instantního droždí}
  \ingredient{1 a půl lžičky soli} promíchat
  \ingredient{1 a půl hrnku vody} přidat a smíchat na lepivé těsto, nechat kynout 12 hodin, vyklopit na pomoučený vál, nožem rozdělit na 8 housek, dát na plech na pečící papír, nechat 30 minut kynout
  \ingredient{slunečnicová semínka} namočit a posypat jimi vykynuté housky, péct na 200°C asi 20-30 minut
\end{recipe}
\newpage

\vspace*{\fill}
\hfill {\Huge\it Dezerty}\hfill
\vspace*{\fill}
\newpage

\begin{recipe}{Bábovka bez tuku s jablky}
  \ingredient{4 vejce}
  \ingredient{200g cukru} utřít
  \ingredient{200g mouky}
  \ingredient{prášek do pečiva} vmíchat
  \ingredient{250g jablek} nastrouhat, vmíchat
  \ingredient{vanilkový cukr} vmíchat, vysypat formu, péct
\end{recipe}
\newpage

\begin{recipe}{Jablečný koláč s drobenkou}
  \ingredient{180 g másla} nechat změknout
  \ingredient{3 vejce}
  \ingredient{200 g cukru krupice} s máslem ušlehat do pěny
  \ingredient{250 g polotučného tvarohu} přidat a smíchat na hladkou směs
  \ingredient{250 g polohrubé mouky}
  \ingredient{1 prášek do pečiva} zapracovat do směsi, těsto nalít na plech
  \ingredient{8 jablek} nastrouhat a položit na těsto
  \ingredient{hrubá mouka}
  \ingredient{cukr krupice}
  \ingredient{máslo} udělat drobenku, nasypat na jablka a péct asi 45 minut na 180°C
\end{recipe}
\newpage

\begin{recipe}{Jáhelné řezy s kokosem}
  \ingredient{180 g jáhel} propláchnout studenou vodou, dvakrát spařit, vařit bez míchání v 500 ml vody 15 - 20 minut
  \ingredient{40 g rozinek} zalít vroucí vodou, nechat nabobtnat
  \ingredient{100 g strouhaného kokosu}
  \ingredient{50 g třtinového cukru}
  \ingredient{15 g vanilkového cukru}
  \ingredient{35 g másla}
  \ingredient{160 g ovesného nebo rýžového mléka} přidat jáhly a rozinky, spojit ve vláčnou hmotu, rozprostřít do formy a zapékat 20 minut při 160 - 170°C
\end{recipe}
\newpage

\begin{recipe}{Jogurtový moučník}
  \ingredient{1 jogurt}
  \ingredient{1 kelímek polohrubé mouky}
  \ingredient{1/2 kelímku cukru}
  \ingredient{1/2 prášku do pečiva} smíchat sypké ingredience
  \ingredient{3 lžíce oleje}
  \ingredient{1 vejce} přidat, smíchat, nalít do dortové formy
  \ingredient{ovoce} poklást na těsto
  \ingredient{hrubá mouka}
  \ingredient{cukr krupice}
  \ingredient{máslo} udělat drobenku, nasypat na ovoce a péct asi 20 minut na 200°C
\end{recipe}
\newpage

\begin{recipe}{Kokosové řezy}
  \ingredient{2 hrnky polohrubé mouky}
  \ingredient{1/2 hrnku moučkového cukru}
  \ingredient{1 prášek do pečiva}
  \ingredient{2 vejce}
  \ingredient{1 hrnek mléka}
  \ingredient{1/2 hrnku rostlinného oleje} smíchat v těsto, nalít na plech
  \ingredient{1 hrnek kokosu}
  \ingredient{1/2 hrnku moučkového cukru} smíchat, nasypat na těsto, péct cca 30 minut na 170°C
  \ingredient{1 šlehačka} nalít na čerstvě upečený moučník
\end{recipe}
\newpage

\begin{recipe}{Makové koláče}
  \ingredient{mléko}
  \ingredient{cukr}
  \ingredient{1/2 kostky droždí}
  \ingredient{polohrubá mouka} zadělat kvásek
  \ingredient{400g polohrubé mouky}
  \ingredient{50g cukru}
  \ingredient{špetka soli}
  \ingredient{60g másla}
  \ingredient{2 žloutky}
  \ingredient{200 ml mléka} zadělat těsto, nechat kynout
  \ingredient{2.5 až 3 dl mléka} dát vařit
  \ingredient{150g mletého maku}
  \ingredient{50g cukru} přidat, vařit 25 minut
  \ingredient{piškotové drobečky} přidat, pokud je nádivka řídká
  \ingredient{hrst nasekaných ořechů}
  \ingredient{skořice}
  \ingredient{rum} dochutit nádivku, vyválet koláč, potřít nádivkou, upéct
\end{recipe}
\newpage

\begin{recipe}{Makovec}
  \ingredient{40dkg polohrubé mouky}
  \ingredient{10dkg cukru moučka}
  \ingredient{20dkg mletého máku}
  \ingredient{5dkg strouhaných vařených brambor}
  \ingredient{přášek do pečiva}
  \ingredient{1 vejce}
  \ingredient{mléko} zadělat hustší těsto
  \ingredient{skořice}
  \ingredient{citronová kůra}
  \ingredient{rum} přidat do těsta, zamíchat, dát do formy, péct, nechat vychladnout, rozříznout
  \ingredient{zavařenina} namazat
  \ingredient{citronová poleva} polít
\end{recipe}
\newpage

\begin{recipe}{Makový závin z bramborového těsta}
  \ingredient{350g polohrubé mouky}
  \ingredient{250g vařených brambor}
  \ingredient{40g hery}
  \ingredient{1 vejce}
  \ingredient{1 žloutek}
  \ingredient{necelý sáček sušeného droždí}
  \ingredient{mléko}
  \ingredient{sůl} zadělat těsto, nechat hodinu kynout
  \ingredient{2.5--3dl mléka}
  \ingredient{150g mletého máku}
  \ingredient{50g cukru} svařit 20 minut
  \ingredient{piškotové drobečky} přidat pokud je řídké
  \ingredient{špetka skořice} přidat
  \ingredient{hrsat nasekaných vlašských ořechů} přidat
  \ingredient{lžíce rumu} přidat, vyválet těsto, dát na něj nádivku, srolovat
  \ingredient{máslo} pomazat pečicí papír, zabalit závin, péct 35 minut při 165 stupních
\end{recipe}
\newpage

\begin{recipe}{Mandlové obdélníky s broskvemi}
  \ingredient{270 g listového těsta} rozválet na dva dlouhé obdélníky (cca o 4 cm širší než velikost broskví)
  \ingredient{1 bílek} potřít okraje vyváleného těsta a asi 1 cm zahnout dovnitř
  \ingredient{35 ks dětských piškotů} rozdrtit
  \ingredient{150 g mandlí} rozsekat a dát trochu bokem na posypání broskví
  \ingredient{2 vanilkové cukry}
  \ingredient{1/4 lžičky citronové kůry}
  \ingredient{40 g másla} smíchat vše kromě těsta a odložených mandlí, navrstvit směs na těsto
  \ingredient{6 větších broskví} oloupat, nakrájet a poklást na směs
  \ingredient{trocha marmelády} poředit a potřít broskve, posypat odloženými mandlemi
  \ingredient{1 vejce} rozšlehat a potřít okraj těsta, péct na 190°C asi 21-23 minut
\end{recipe}
\newpage

\begin{recipe}{Medovník}
  \ingredient{180 g másla}
  \ingredient{180 g moučkového cukru}
  \ingredient{1 vejce}
  \ingredient{6 zarovnaných lžic medu}
  \ingredient{4 lžíce smetany} šlehat v misce nad párou, krém nebude hustý
  \ingredient{450 g hladké mouky}
  \ingredient{1,5 lžičky jedlé sody} smíchat, přidat teplý krém a vypracovat těsto, přikryté nechat 5 minut odpočinout, z kousků těsta cca 19 dkg na pečicím papíře vyválet placky 2-3 mm silné a vykrojit kola o průměru 22 cm, péct na 180°C cca 4 minuty, na závěr upéct zbytky těsta
  \ingredient{1 sladké zahuštěné kondenzované mléko (397 g)} vařit 2 hodiny ponořené ve vodě, nechat zchládnout
  \ingredient{70 g mletých vlašských ořechů}
  \ingredient{200 g másla} utřít spolu se zkaramelizovaným mlékem, nenechat ztuhnout
  \ingredient{2,5 lžíce cukru krystal}
  \ingredient{1,5 dcl převařené vody}
  \ingredient{50 ml rumu} smíchat na sirup, postupně každou placku potřít zespoda sirupem (kromě spodní, na každou plochu cca 20 ml sirupu), přiložit na dort, potřít sirupem i zvrchu a natřít na ni krém, na závěr potřít i vrch a strany dortu
  \ingredient{30 g vlašských ořechů} nasekat a smíchat s 50 g pečeného zbytku těsta, posypat strany a vršek medovníku, medovník nechat 24 hodin uležet
\end{recipe}
\newpage

\begin{recipe}{Minijablka v županu}
  \ingredient{25dkg mouky}
  \ingredient{25dkg tuku} smíchat s čtvrtinou mouky, vypracovat těsto, dát do chladu 
  \ingredient{6 lžic vody}
  \ingredient{špetka soli} rozmíchat ve vodě
  \ingredient{1 lžíce 4\% octa} rozmíchat ve vodě, nalít do zbytku mouky
  \ingredient{žloutek} do mouky, vypracovat těsto, nechat půl hodiny odpočinout, obě těsta vyválet, dát tukové přes
    závinové a zabalit do něj, rozválet a složit třetiny podélně i příčně, nechat alespoň půl hodiny (radši přes noc)
    odpočinout, vyválet a složit na třetiny podélně i příčně, rozválet
  \ingredient{jablka} ukrojit kousky
  \ingredient{sokřicový cukr} obalit jablka
  \ingredient{čokoláda} na těsto, na to jablka a zabalit do šátečku, upéct
\end{recipe}
\newpage

\begin{recipe}{Míša dort bez výčitek}
  \ingredient{300 g ovesných vloček} rozmixujeme na jemnou mouku
  \ingredient{30 g kakaa} 
  \ingredient{50 g kokosu}
  \ingredient{8 g jedlé sody} promícháme s moukou
  \ingredient{3 lžíce olivového oleje}
  \ingredient{3 lžíce medu}
  \ingredient{350 ml mléka} přimícháme k sypkým ingrediencím, vznikne spojité těsto
  \ingredient{5 vaječných bílků}
  \ingredient{špetka soli} vyšlehat hustý sníh a jemně vmíchat do těsta
  \ingredient{kapka oleje} vymazat dortovou formu
  \ingredient{trocha kokosu} vysypat formu, vlít těsto a péct asi 45 minut na 180°C
  \ingredient{750 g nízkotučného tvarohu}
  \ingredient{40 g kokosu}
  \ingredient{4 lžíce medu} smícháme a dáme do lednice, krémem přikryjeme upečený a vychladlý dort
  \ingredient{čokoláda nebo ovoce} ozdobit
\end{recipe}
\newpage

\begin{recipe}{Muffiny - základní recept}
  \ingredient{250 g hladké mouky (na ovocné 200 g hladké a 50 g polohrubé)}
  \ingredient{100 g cukru} 
  \ingredient{2 lžičky prášku do pečiva}
  \ingredient{vanilkový cukr}
  \ingredient{špetka soli} smíchat sypké přísady
  \ingredient{85 g másla} rozpustit a nechat zchladnout
  \ingredient{2 vejce}
  \ingredient{200 ml mléka} smíchat tekuté přísady a přidat k sypkým, naplnit formičky a péct na 180°C asi 20-25 minut
\end{recipe}
\newpage

\begin{recipe}{Muffiny jablečné se skořicovou drobenkou}
  \ingredient{suroviny ze základního receptu}
  \ingredient{2 lžičky mleté skořice} přidat k sypkým přísadám
  \ingredient{4 jablka} přidat k tekutým přísadám
  \ingredient{máslo}
  \ingredient{cukr}
  \ingredient{hrubá mouka}
  \ingredient{skořice} vypracovat drobenku a nasypat na muffiny
\end{recipe}
\newpage

\begin{recipe}{Muffiny mandlovo-banánové}
  \ingredient{suroviny ze základního receptu}
  \ingredient{nasekaná vysokoprocentní čokoláda}
  \ingredient{hrst spařených sušených brusinek} přidat k sypkým přísadám
  \ingredient{2 (pře)zralé banány}
  \ingredient{lžička mandlové esence} přidat k tekutým přísadám
\end{recipe}
\newpage

\begin{recipe}{Muffiny pomerančové s brusinkami a polevou}
  \ingredient{suroviny ze základního receptu} mléka dát jen polovinu
  \ingredient{kůra z jednoho pomeranče}
  \ingredient{hrst spařených sušených brusinek} přidat k sypkým přísadám
  \ingredient{100 ml čerstvě vymačkané pomerančové šťávy} přidat k tekutým přísadám
  \ingredient{100 g moučkového cukru}
  \ingredient{špetka skořice}
  \ingredient{lžíce pomerančové šťávy} prošlehat metličkou, vychladlé muffiny namáčet v polevě
\end{recipe}
\newpage

\begin{recipe}{Muffiny třešňové s bílou čokoládou}
  \ingredient{suroviny ze základního receptu}
  \ingredient{2 lžíce holandského kakaa}
  \ingredient{150 g nadrobno nakrájené čokolády (půl tmavé a půl bílé)}
  \ingredient{hrst vypeckovyných třešní pokrájených na menší kousky} přidat k sypkým přísadám
\end{recipe}
\newpage

\begin{recipe}{Nepečený moučník z pudinku a tvarohu}
  \ingredient{500 ml mléka}
  \ingredient{cukr dle chuti}
  \ingredient{2 balení pudinku} uvařit, nechat vychladnout
  \ingredient{500 g tvarohu} přimíchat k pudinku
  \ingredient{kokos} vysypat dortovou formu, přidat část směsi
  \ingredient{dětské piškoty} položit na směs
  \ingredient{černá káva} pokapat piškoty, pak přidat opět směs, případně další vrstvy, na závěr směs a zasypat kokosem
\end{recipe}
\newpage

\begin{recipe}{Ovocný jogurt}
  \ingredient{1l ovoce} rozmačkat
  \ingredient{cukr} přidat
  \ingredient{jogurt} vmíchat
\end{recipe}
\newpage

\begin{recipe}{Pečený rebarborový kompot}
  \ingredient{700g rebarbory} nakrájet, oloupat, do misky
  \ingredient{80g cukru} přimíchat, přikrýt alobalem, dát péct na 200 stup\-ňů dvacet minut, pak dopéct bez alobalu pět minut
  \ingredient{šlehačka nebo mascarpone} navrch
\end{recipe}
\newpage

\begin{recipe}{Podzimní hruškový koláč}
  \ingredient{200 g hladké mouky}
  \ingredient{1 lžíce moučkového cukru}
  \ingredient{1 vanilkový cukr} 
  \ingredient{špetka soli} smíchat
  \ingredient{100 g změklého másla}
  \ingredient{1 vejce} přidat k mouce, vypracovat hladké těsto, nechat v chladu půl hodiny odpočinout, pak vyválet nebo vmačkat do kulaté formy, kraje těsta vytáhnout i do boků
  \ingredient{2 vejce} oddělit žloutky a bílky
  \ingredient{4 lžíce cukru krupice}
  \ingredient{1 vanilkový cukr} utřít se žloutky do pěny
  \ingredient{500 g měkkého tvarohu} přidat, vymíchat dohladka a zlehka přimíchat sníh z bílků, náplň rovnoměrně rozetřít na syrový korpus
  \ingredient{2 - 3 hrušky (nebo jablka)} oloupat, rozkrájet na poloviny, odstranit jádřince a vějířovitě nakrojit
  \ingredient{3 lžíce citronové šťávy} pokapat na hruškové vějířky, vějířky naskládat na tvarohovou směs
  \ingredient{skořicový cukr} posypat hrušky, koláč péct na 170°C do růžova
\end{recipe}
\newpage

\begin{recipe}{Ptáčci z kynutého těsta}
  \ingredient{mléko}
  \ingredient{30g droždí}
  \ingredient{cukr} nechat vzejít kvásek
  \ingredient{500g polohrubé mouky} prosát, přidat kvásek
  \ingredient{100g másla} rozehřát, zchladlé přidat
  \ingredient{3 žloutky}
  \ingredient{80g cukru}
  \ingredient{1/4l mléka}
  \ingredient{vanilkový cukr}
  \ingredient{citronová kůra}
  \ingredient{sůl} zadělat těsto, nechat půl hodiny kynout, prohníst, nechat hodinu kynout, udělat uzlíky, upéct
\end{recipe}
\newpage

\begin{recipe}{Rybízová klasika}
  \ingredient{100 ml vlažného mléka}
  \ingredient{35 g droždí}
  \ingredient{35 cukru} zadělat kvásek
  \ingredient{500 g polohrubé mouky}
  \ingredient{40 g cukru}
  \ingredient{kůra z 1 citronu}
  \ingredient{lžička soli}
  \ingredient{2 žloutky}
  \ingredient{75 g změklého másla}
  \ingredient{cca 150 ml mléka} přidat kvásek, zadělat těsto a nechat kynout na teplém místě na dvojnásobný objem, rozdělit na poloviny na 2 plechy, rukama těsto roztáhnout a roztlačit po celé ploše
  \ingredient{250 g polotučného tvarohu v kostce}
  \ingredient{1 vejce} 
  \ingredient{2 lžíce cukru krupice}
  \ingredient{dle potřeby mléko} utřít dohladka, zdobicím sáčkem udělat střídavě asi 3 cm široké pruhy tvarohu a prázdného místa
  \ingredient{500 g rybízu} nasypat do vzniklých mezer
  \ingredient{100 g másla}
  \ingredient{140 g hrubé mouky}
  \ingredient{100 g cukru krupice} udělat drobenku, posypat hlavně rybízové pruhy
  \ingredient{1 vejce} potřít okraje koláčů, péct asi 20 minut na 170°C, dokud okraje nezezlátnou
\end{recipe}
\newpage

\begin{recipe}{Slovenský měkký perník}
  \ingredient{400g polohrubé mouky}
  \ingredient{250g cukru krupice}
  \ingredient{2 lžíce kakaa}
  \ingredient{lžička skořice}
  \ingredient{kypřící prášek do perníku}
  \ingredient{ořrchy nebo kandované ovoce} smíchat
  \ingredient{2 vejce}
  \ingredient{2 lžíce medu}
  \ingredient{100ml oleje}
  \ingredient{500ml mléka} smíchat, zadělat těsto, péct pří 180--200 stupních
\end{recipe}
\newpage

\begin{recipe}{Smažená jablka}
  \ingredient{4 velká jablka} oloupat, nakrájet na silné plátky, vykrojit jádřinec
  \ingredient{citronová šťáva}
  \ingredient{rum} nakapat na jablka
  \ingredient{vanilkový cukr}
  \ingredient{skořice} nasypat na jablka
  \ingredient{1/8 l mléka}
  \ingredient{špetka soli}
  \ingredient{2 žloutky}
  \ingredient{100g mouky} připravit těstíčko
  \ingredient{2 bílky} ušlehat sníh
  \ingredient{40g cukru} zašlehat do sněhu, sníh vmíchat do těstíčka, obalit jablka, smažit je
\end{recipe}
\newpage

\begin{recipe}{Švestkový táč na plech}
  \ingredient{30g droždí}
  \ingredient{lžíce cukru}
  \ingredient{lžíce hladké mouky}
  \ingredient{100 ml vlažného mléka} zadělat kvásek
  \ingredient{500g mouky}
  \ingredient{200g cukru}
  \ingredient{špetka soli} smíchat
  \ingredient{400 ml mléka} přilít
  \ingredient{2 vejce} rozkvedlat, přilít
  \ingredient{100g rozpuštěného másla} přilít, přidat kvásek, nechat kynout, až bude dvoj\-ná\-sob\-né,
    důrazně míchat, vyhnat vzduch, nalít na plech a půl hodiny nechat kynout 
  \ingredient{50g másla}
  \ingredient{50g mletých vlašských ořechů}
  \ingredient{50g hrubé mouky}
  \ingredient{50g moučkového cukru}
  \ingredient{1/2 lžičky mleté skořice} udělat drobenku
  \ingredient{švestky} vypeckovat, dát na koláč, posypat drobenkou dát péct (10 minut 180, pak 160)
\end{recipe}
\newpage

\begin{recipe}{Třešňový koláč s pudinkem a drobenkou}
  \ingredient{200 g polohrubé mouky}
  \ingredient{5 - 6 lžic moučkového cukru}
  \ingredient{1/2 prášku do pečiva}
  \ingredient{75 ml mléka}
  \ingredient{1 vejce}
  \ingredient{4 lžíce oleje} zpracovat v hladké těsto, rozetřít na plech
  \ingredient{vypeckované třešně} hustě poklást na těsto
  \ingredient{3/4 l mléka}
  \ingredient{2 balíčky vanilkového pudinku}
  \ingredient{cukr dle chuti} uvařit a ještě teplý (ne vařící) vylít na třešně
  \ingredient{100 g polohrubé mouky}
  \ingredient{50 g moučkového cukru}
  \ingredient{1/2 balíčku vanilkového cukru}
  \ingredient{50 g Hery} udělat drobenku, nasypat na pudink, péct na 200°C asi 30 - 35 minut
\end{recipe}
\newpage

\begin{recipe}{Tvarohový koláč s malinovou omáčkou}
  \ingredient{100 g cukru}
  \ingredient{150 g hladké mouky}
  \ingredient{1/2 prášku do pečiva}
  \ingredient{100 g másla}
  \ingredient{1 vejce}
  \ingredient{citronová kůra} vypracovat těsto a dát odpočinout do lednice, pak vmačkat do formy na tloušťku cca 4 mm
  \ingredient{500 g tvarohu}
  \ingredient{2 vejce}
  \ingredient{vanilkový pudink}
  \ingredient{lžíce rumu}
  \ingredient{175 g cukru} smíchat, směs nalít na korpus a péct na 170°C asi 30 - 40 minut
  \ingredient{maliny}
  \ingredient{cukr}
  \ingredient{trocha vody} vařit 15 minut, pak scedit přes cedník a svařit do hustší konzistence
\end{recipe}
\newpage

\begin{recipe}{Vánočka}
  \ingredient{mléko}
  \ingredient{mouka}
  \ingredient{cukr}
  \ingredient{84g droždí} zadělat kvásek
  \ingredient{200g másla} rozehřát a nechat zchládnout
  \ingredient{1 kg polohrubé mouky}
  \ingredient{250g cukru}
  \ingredient{4 žloutky}
  \ingredient{kůra z jednoho citronu} spolu s kváskem a máslem zadělat těsto
  \ingredient{140g mandlí}
  \ingredient{140g rozinek} do těsta, nechat kynout, uplést vánočku, nechat kynout
  \ingredient{žloutak} potřít vánočku
  \ingredient{mandle} posypat vánočku
  \ingredient{} péct, když začne zlátnout, zakrýt papírem
\end{recipe}
\newpage

\vspace*{\fill}
\hfill {\Huge\it Cukroví}\hfill
\vspace*{\fill}
\newpage

\begin{recipe}{Marcipán bechyňský}
  \ingredient{500g (+250g) hladké mouky}
  \ingredient{300g cukru}
  \ingredient{skořice}
  \ingredient{hřebíček}
  \ingredient{kůra ze dvou citronů}
  \ingredient{9g sody}
  \ingredient{200g medu}
  \ingredient{3 vejce} zadělat těsto, vyválet placku, vykrájet tvary, péct
  \ingredient{žloutek} potřít upečené perníky
\end{recipe}
\newpage

\begin{recipe}{Perníčky s anýzem}
  \ingredient{250g hladké mouky}
  \ingredient{250g žitné mouky}
  \ingredient{120g moučkového cukru}
  \ingredient{lžička mletého anýzu}
  \ingredient{lžička skořice}
  \ingredient{1 tlučený hřebíček}
  \ingredient{lžička jedlé sody}
  \ingredient{120g Hery}
  \ingredient{2 vejce}
  \ingredient{3 lžíce medu} udělat těsto, nechat den odležet, rozválet a vy\-krá\-jet, upéct
  \ingredient{vajíčko} rozkvedlat, upečené hned potřít, nechat dva dny
  \ingredient{14dkg moučkového cukru}
  \ingredient{bílek}
  \ingredient{citronová šťáva} udělat polevu, ozdobit perníčky
  \ingredient{mletý anýz} posypat
\end{recipe}
\newpage

\begin{recipe}{Placičky z bílého máku}
  \ingredient{250g rostlinného tuku}
  \ingredient{170g moučkového cukru} ušlehat do pěny
  \ingredient{1 vejce} zašlehat
  \ingredient{vanilkový cukr} zašlehat
  \ingredient{150g polohrubé mouky} zašlehat
  \ingredient{80g bílého máku} zašlehat
  \ingredient{130g škrobové moučky} zašlehat, dát kopičky na plech
  \ingredient{modrý mák} posypat, péct při 150 stupních
\end{recipe}
\newpage

\begin{recipe}{Rakvičky}
  \ingredient{4 žloutky}
  \ingredient{1 vejce}
  \ingredient{300 g moučkového cukru} šlehat metličkami na nejvyšší stupeň 2 minuty
  \ingredient{3 polévkové lžíce polohrubé mouky}
  \ingredient{1/4 lžičky jedlé sody} smíchat a zapracovat do hmoty, plnit do máslem vymazaných formiček asi do čtvrtiny, péct asi 50 minut na 80°C
\end{recipe}
\newpage

\begin{recipe}{Sušenky}
  \ingredient{40 dkg mouky} do mísy
  \ingredient{10 dkg másla/sádla} rozsekat do mouky
  \ingredient{12 dkg cukru} přidat
  \ingredient{2 vejce} přidat
  \ingredient{1/2 balíčku prášku do pečiva} přidat, vypracovat těsto, vyválet, vykrajovat tva\-ry, péct do růžova
\end{recipe}
\newpage

\begin{recipe}{Valentýnská srdíčka}
  \ingredient{300g polohrubé mouky}
  \ingredient{250g mletého máku}
  \ingredient{210g moučkového cukru}
  \ingredient{2 žloutky}
  \ingredient{250g másla nebo Hery} zadělat těsto, vykrájet tvary a péct při teplotě 180 stupňů
  \ingredient{pikantní maremeláda} případně potřít a slepit
\end{recipe}
\newpage

\begin{recipe}{Vimperská máčata}
  \ingredient{300g bílého máku} umlít
  \ingredient{200g jader z vlašských ořechů} nastrouhat
  \ingredient{200g cukru} smíchat
  \ingredient{rum} přidat
  \ingredient{3 -- 4 vejce} přidat, umíchat těstíčko, vytvarovat kuličky nebo placku
  \ingredient{čokoládová poleva} rozhřát, kuličky namočit nebo placku polít
\end{recipe}
\newpage

\vspace*{\fill}
\hfill {\Huge\it Nápoje}\hfill
\vspace*{\fill}
\newpage

\begin{recipe}{Citronáda}
  \ingredient{šťáva ze 4 citronů}
  \ingredient{1l vody}
\end{recipe}
\newpage

\begin{recipe}{Domácí Baileys}
  \ingredient{2 plechovky Salka} zavřené vařit 2 hodiny ve vodě, před otevřením nechat zchladnout
  \ingredient{500 ml mléka}
  \ingredient{1 balíček mandlového pudinku}
  \ingredient{1 vanilkový cukr} za stálého míchání připravit pudink
  \ingredient{500 ml rumu} spolu se zkaramelizovaným Salkem přimíchat k vlažnému pudinku
  \ingredient{100 ml šlehačky} přimíchat do směsi, nechat minimálně 12 hodin uležet v lednici
\end{recipe}
\newpage

\begin{recipe}{Jahodové lassi}
  \ingredient{100g jahod}
  \ingredient{100g bílého jogurtu}
  \ingredient{lžíce medu}
  \ingredient{šťáva z 1/8 citronu}
  \ingredient{100ml studené vody}
  \ingredient{3 kostky ledu} rozmixovat
\end{recipe}
\newpage

\begin{recipe}{Ovesné mléko}
  \ingredient{300 ml vody} uvařit
  \ingredient{pár zrnek soli}
  \ingredient{3 lžíce ovesných vloček} zalít vroucí vodou, nechat hodinu stát, pak chvíli povařit, rozmixovat, přecedit a zchladit
\end{recipe}
\newpage

\vspace*{\fill}
\hfill {\Huge\it Těsta}\hfill
\vspace*{\fill}
\newpage

\begin{recipe}{Kynuté těsto na taštičky a jiné}
  \ingredient{50 ml mléka}
  \ingredient{1/2 kostky droždí}
  \ingredient{trocha cukru} zaděláme kvásek
  \ingredient{1/2 kg hladké mouky}
  \ingredient{150 ml mléka}
  \ingredient{150 ml oleje}
  \ingredient{špetka soli} přidáme kvásek, promícháme a necháme kynout, můžeme plnit zelím, sýrem, pórkem, tvarohem, čokoládou..., lze péct, smažit i vařit
\end{recipe}
\newpage

\end{document}
\endinput
%%
%% End of file `kraut.tex'.
